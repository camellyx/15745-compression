\begin{abstract}

The huge growing performance gap between modern processors and memory ( i.e. the
so called memory wall) is a significant barrier that limits system performance
and hardware caches help mitigate this problem. Sophisticated techniques are
needed now more than ever to better utilize the cache. Cache compression is a
technique that can increase effective cache capacity and reduce misses, improve
performance, and potentially reduce system energy. Hardware compression
algorithms need to be simple in order to reduce the performance overhead as well
as the implementation complexity. The effectiveness of these algorithms are
limited by the layout and range of data in memory. 

The aim of this project is to implement and evaluate compiler optimizations to
improve compressibility of applications’ data in the memory hierarchy.
Specifically we evaluate data structure splitting as a compiler technique to
place data with low relative dynamic range contiguously in memory, thereby
increasing the potential compression ratio when these algorithms are employed.
In this project we implement data structure splitting in LLVM and evaluate the
benefits of our optimization using custom benchmarks. We also explore the
performance impact of this optimization for applications with different data
structure types and memory access patterns and the interaction between cache
locality and data compressibility.

\end{abstract}
