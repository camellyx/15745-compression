\section{Methodology and Implementation}
\label{sec:meth}

There are three common methods to split structures [1] : { \em affinity based
splitting},  { \em frequency-based splitting} and \em {maximal splitting}. {\em
Affinity-based splitting} typically requires a profiling run to analyze and
determine the affinity of the fields in a structure. Ideally the high affinity
fields should be grouped together while splitting, to ensure that there is no
loss in locality benefits. The structure is then broken down based on the field
affinity. {\em Frequency-based splitting} also needs information about how often
each field is accessed and this can be obtained from a profile or memory trace.
The fields are grouped into frequently-accessed fields, known as hot fields and
infrequently-accessed fields, or cold fields. The structure is split to separate
the {\em hot} and {\em cold fields}. {\em Maximal splitting} does not group any
of the fields in a structure and completely separates all the fields. We
evaluate the impact of splitting and compression with different access patterns,
frequencies (hot vs. cold fields) across a range of cache locality. Due to time
and space constraints we only evaluate maximal splitting. 

Our optimization algorithm functions as follows:

\squishlist

\item Identify all instructions creating arrays of structs.

\item Check if the the structure should be split (currently we use an input file).
Identify the fields in the struct and store the fields and the instruction
generating the fields in a dictionary.

\item Traverse the instruction list, and identify accesses to the data structure. In
LLVM, data structures are accessed using pointers. Some of these are
‘intermediate’ pointers and others are ‘leaf’  pointers which actually point to
the data being accessed.

\item Create a tree from different levels of intermediate pointers and corresponding
leaf pointers.

\item Traverse the pointer tree and compute the corresponding pointers to the new
split arrays.

\item Replace each instruction that generates a pointer with a newly created
instruction that generates the new corresponding pointer as determined above.

\item Replace all uses of a leaf pointer with that of the new leaf pointer
instruction.

\item Traverse the pointer dependence tree above in reverse order and delete all the
old instructions. This is done to ensure that we do not delete an instruction
while it is still being used.

\squishend

Our implementation algorithm ensures that all pointer references at any level of
the overall structure will be redirected to the new split arrays correctly. We
also coalesce multiple pointer computations to reduce the number of these
instructions in generated program. The framework also allows for multiple levels
of nesting with the data structure and enables splitting structures allocated in
the stack, heap as well as global structures. We, however, limit our evaluation
to data structures with a single level of nesting. 
