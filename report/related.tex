\section{Related Works}
\label{sec:rel}

Gennady Pekhimenko et. al proposed Base-Delta-Immediate Compression~\cite{bdi}. The key
idea in Base-Delta-Immediate Compression is that data in any cache frequently
have low dynamic range and hence can be compressed with a common base and small
variations or deltas from this base. The authors observe that for many cache
lines, multiple bases are required for efficient compressing due to multi-field
data structures. Trishul Chilimbi et al. proposed Cache-Conscious Structure
Layout~\cite{cache-layout}. The key idea in Cache-Conscious Structure Layout is that by coloring
and clustering pointers, pointers with spatial locality are grouped together
into a single cache line which improves data reuse. Trishul Chilimbi et al. take
the concept further in Cache-Conscious Structure Definition~\cite{cache-def}. In
Cache-Conscious Structure Definition, fields within a structure are further
split into different fields which are reordered to further increase cache
locality and data reuse. Recently, Chris Lattner et al. proposed Automatic Pool
Allocation~\cite{lattner1}, which segregates heap-based data into separate memory pools and
controls the internal data layout with heuristics. The key idea is to analyze
heap allocated data with a points-to graph and a call graph. Grouping data with
spatial locality together, improves performance. Finally, Stephen Curial
proposed MPADS: Memory-Pooling-Assisted Data Splitting~\cite{mpads}, which analyzes
affinity (locality) between fields in a structure and splits data fields based
on their affinity to reduce memory footprint and improve application’s
performance. Fields with low affinity are separated into different pools while
fields with high affinity are bundled together at the granularity of each
element. Memory-Pooling-Assisted Data Splitting improves data reuse by pooling
fields with high affinity while improving memory efficiency by splitting
isolated fields.
