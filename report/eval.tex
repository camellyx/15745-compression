\section{Evaluation}
\label{sec:eval}
We evaluate the benefits of our implementation on our micro-benchmark (described above) across a range of parameters. We implement the compression algorithm in the last level of cache. The metrics used for evaluation are the miss rate in the last level of cache in MPKI and the effective cache capacity increase or effective compression ratio (ECR) which measures the compressibility of the data that can actually be utilized with the limitations of hardware implementation. Figure .. shows the benefits of the splitting optimization for different data set sizes. Figure .. illustrates the interaction between locality and splitting. Here there are two different access patterns - streaming and random accesses. With a streaming access pattern the.... The splitting benefit for different field affinities is depicted in figure .. Here we vary the probability that two fields are accessed together. In this case we use a random access pattern. There is no change in compressibility, as expected. As the affinity between the two fields increases, the miss rate increases when data splitting is employed. This is because, the benefits from locality outweigh the compression benefits for all cases except when there is little or no affinity between the fields. The effect is further exacerbated by the lack of reuse in the data since the access pattern is random. And finally, we evaluate the impact of our optimization for different field types. We use different combinations of char, int, float and pointer and illustrate the difference in compressibility and cache utilization when using data splitting across different field types. Figure shows that the impact of data splitting is greatest with disparate data fields like ... and ... in the same struct. 
